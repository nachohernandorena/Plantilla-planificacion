\documentclass[
11pt, % The default document font size, options: 10pt, 11pt, 12pt
%codirector, % Uncomment to add a codirector to the title page
]{charter} 




% El títulos de la memoria, se usa en la carátula y se puede usar el cualquier lugar del documento con el comando \ttitle
\titulo{Sistema de gestión remota de ganado} 

% Nombre del posgrado, se usa en la carátula y se puede usar el cualquier lugar del documento con el comando \degreename
%\posgrado{Carrera de Especialización en Sistemas Embebidos} 
\posgrado{Carrera de Especialización en Internet de las Cosas} 
%\posgrado{Carrera de Especialización en Intelegencia Artificial}
%\posgrado{Maestría en Sistemas Embebidos} 
%\posgrado{Maestría en Internet de las cosas}

% Tu nombre, se puede usar el cualquier lugar del documento con el comando \authorname
\autor{Ing. Ignacio Pablo Hernandorena} 

% El nombre del director y co-director, se puede usar el cualquier lugar del documento con el comando \supname y \cosupname y \pertesupname y \pertecosupname
\director {Mg. Ing. Carlos Canal (pendiente aprobación)}
\pertenenciaDirector{UNSAM} 
% FIXME:NO IMPLEMENTADO EL CODIRECTOR ni su pertenencia
%\codirector{John Doe} % para que aparezca en la portada se debe descomentar la opción codirector en el documentclass
%\pertenenciaCoDirector{FIUBA}

% Nombre del cliente, quien va a aprobar los resultados del proyecto, se puede usar con el comando \clientename y \empclientename
\cliente{-}
\empresaCliente{Empresas o personas en el rubro ganadero}

% Nombre y pertenencia de los jurados, se pueden usar el cualquier lugar del documento con el comando \jurunoname, \jurdosname y \jurtresname y \perteunoname, \pertedosname y \pertetresname.
\juradoUno{Nombre y Apellido (1)}
\pertenenciaJurUno{pertenencia (1)} 
\juradoDos{Nombre y Apellido (2)}
\pertenenciaJurDos{pertenencia (2)}
\juradoTres{Nombre y Apellido (3)}
\pertenenciaJurTres{pertenencia (3)}
 
\fechaINICIO{25 de abril de 2023}		%Fecha de inicio de la cursada de GdP \fechaInicioName
\fechaFINALPlan{13 de junio de 2023} 	%Fecha de final de cursada de GdP
\fechaFINALTrabajo{15 de mayo de 2024}	%Fecha de defensa pública del trabajo final


\begin{document}

\maketitle
\thispagestyle{empty}
\pagebreak


\thispagestyle{empty}
{\setlength{\parskip}{0pt}
\tableofcontents{}
}
\pagebreak


\section*{Registros de cambios}
\label{sec:registro}


\begin{table}[ht]
\label{tab:registro}
\centering
\begin{tabularx}{\linewidth}{@{}|c|X|c|@{}}
\hline
\rowcolor[HTML]{C0C0C0} 
Revisión & \multicolumn{1}{c|}{\cellcolor[HTML]{C0C0C0}Detalles de los cambios realizados} & Fecha      \\ \hline
0      & Creación del documento                                 &\fechaInicioName \\ \hline
1      & Se completa hasta el punto 5 inclusive                 & 09 de mayo de 2023 \\ \hline
%2      & Se completa hasta el punto 7 inclusive
%		  Se puede agregar algo más \newline
%		  En distintas líneas \newline
%		  Así                                                    & dd/mm/aaaa \\ \hline
%3      & Se completa hasta el punto 11 inclusive                & dd/mm/aaaa \\ \hline
%4      & Se completa el plan	                                 & dd/mm/aaaa \\ \hline
\end{tabularx}
\end{table}

\pagebreak



\section*{Acta de constitución del proyecto}
\label{sec:acta}

\begin{flushright}
Buenos Aires, \fechaInicioName
\end{flushright}

\vspace{2cm}

Por medio de la presente se acuerda con el \authorname\hspace{1px} que su Trabajo Final de la \degreename\hspace{1px} se titulará ``\ttitle'', consistirá esencialmente en la implementación de un prototipo de un sistema de gestión remota de ganado, y tendrá un presupuesto preliminar estimado de \textcolor{red}{600}\ horas de trabajo y \textcolor{red}{\$XX}, con fecha de inicio \fechaInicioName\hspace{1px} y fecha de presentación pública \fechaFinalName.

Se adjunta a esta acta la planificación inicial.

\vfill

% Esta parte se construye sola con la información que hayan cargado en el preámbulo del documento y no debe modificarla
\begin{table}[ht]
\centering
\begin{tabular}{ccc}
\begin{tabular}[c]{@{}c@{}}Dr. Ing. Ariel Lutenberg \\ Director posgrado FIUBA\end{tabular} & \hspace{2cm} & \begin{tabular}[c]{@{}c@{}}\clientename \\ \empclientename \end{tabular} \vspace{2.5cm} \\ 
\multicolumn{3}{c}{\begin{tabular}[c]{@{}c@{}} \supname \\ Director del Trabajo Final\end{tabular}} \vspace{2.5cm} \\
%\begin{tabular}[c]{@{}c@{}}\jurunoname \\ Jurado del Trabajo Final\end{tabular}     &  & \begin{tabular}[c]{@{}c@{}}\jurdosname\\ Jurado del Trabajo Final\end{tabular}  \vspace{2.5cm}  \\
%\multicolumn{3}{c}{\begin{tabular}[c]{@{}c@{}} \jurtresname\\ Jurado del Trabajo Final\end{tabular}} \vspace{.5cm}                                                                     
\end{tabular}
\end{table}




\section{1. Descripción técnica-conceptual del proyecto a realizar}
\label{sec:descripcion}

La tecnología IoT (del inglés \emph{Internet of Things}) ha revolucionado muchas industrias, incluida la agrícola y ganadera. El desarrollo de sistemas de gestión de ganado utilizando IoT ayuda al personal rural a monitorear y controlar sus tierras y animales, proporcionando así condiciones óptimas para el progreso de esta actividad.

Hoy existe una demanda de este tipo de tecnología en Argentina, donde la ganadería es una parte importante de la economía. Algunos de los desafíos comunes que enfrentan los propietarios y cuidadores de ganado es controlar que los animales no deambulen y se pierdan, o que causen daños a los cultivos y a otras propiedades.

Mediante el uso de la tecnología IoT se puede desarrollar un dispositivo que ayude a mantener al ganado en su lugar. Si a esto se le suma la posibilidad de mantener un registro completo de cada animal se obtiene una solución integral para el propietario.

Para abordar esta problemática se propone la implementación de un prototipo que utilizará un sistema IoT para la gestión remota de ganado. Cabe mencionar que este proyecto se trata de una iniciativa personal, con el objetivo de contribuir al avance en el ámbito de la tecnología y la ganadería.

El sistema consta de etiquetas electrónicas (en adelante denominados \emph{TAGs}) que serán fijadas en las orejas de los animales. Para ello, se utilizarán dispositivos de bajo consumo de energía que emplean la tecnología de red de largo alcance conocida como LoRaWAN. Además, se desarrollará una plataforma que permitirá gestionar, controlar, almacenar y visualizar los datos generados por estos dispositivos.

Como se puede observar en la figura \ref{fig:diagBloques}, para implementar este sistema se requieren diversos componentes, como los \emph{TAGs}. Los mismos estarán compuestos de un panel solar, un sistema de alimentación, el GPS (del inglés \emph{Global Positioning System}) y el dispositivo con LoRaWAN que obtendrá los datos y los enviará a un gateway. 

El gateway actuará recopilando y procesando la información de los \emph{TAGs} y enviándola a una base de datos remota. Este dispositivo requerirá una conexión a internet para la vinculación con la plataforma en la nube llamada \emph{The Things Stack}.

\emph{The Things Stack} es un servidor de red LoRaWAN altamente escalable y de código abierto que permite que las aplicaciones y los servicios de IoT conecten y administren dispositivos, datos y aplicaciones de manera segura. 

\begin{figure}[htpb]
\centering 
\includegraphics[width=.7\textwidth]{./Figuras/diagBloques.png}
\caption{Diagrama en bloques del sistema}
\label{fig:diagBloques}
\end{figure}

\emph{The Things Stack} ofrece una amplia gama de servicios para admitir aplicaciones IoT, dentro de los que se incluyen: 
\begin{itemize}
	\item Servidor de gateway: proporciona herramientas para administrar dispositivos y sus datos asociados, incluida la activación, el registro y la autorización de dispositivos.
	\item Servidor de identidad: ofrece características de seguridad avanzadas, incluido el cifrado de extremo a extremo y protocolos de comunicación seguros, para garantizar la privacidad y seguridad de los datos de IoT.
	\item Servidor LoRaWAN: es el encargado de recibir y procesar datos de dispositivos IoT y de administrar la activación, la seguridad y el cifrado de datos del dispositivo. Además permite la recopilación, el almacenamiento y el análisis de datos generados por los dispositivos IoT.
	\item Servidor de registro: es responsable de autorizar de forma segura nuevos dispositivos para conectarse a la red y de administrar el proceso de autenticación de dispositivos.
	\item Servidor de aplicaciones: debe procesar y enrutar los datos de los dispositivos a las aplicaciones o servicios apropiados.
	\item Monitoreo:
		\begin{itemize}
		\item Tablero: consiste en un tablero personalizable que muestra información en tiempo real sobre el rendimiento de la red, la actividad del dispositivo y el uso de datos.
		\item Eventos: se pueden visualizar alertas y notificaciones para eventos clave, como activación de dispositivos, transmisión de datos y problemas de rendimiento de la red.
		\end{itemize}
	\item Integración: ofrece una variedad de opciones de integración para conectarse con otros sistemas, como otras plataformas en la nube, bases de datos y herramientas de análisis. Para este proyecto se realizará una integración mediante HTTP (del inglés \emph{Hypertext Transfer Protocol}) a \emph{AllThingsTalk}.
\end{itemize}

\emph{AllThingsTalk} es una plataforma de IoT basada en la nube que proporciona una variedad de herramientas y servicios para crear, implementar y administrar aplicaciones y dispositivos de IoT. Esta plataforma ofrece una interfaz fácil de usar y un conjunto de componentes básicos personalizables, lo que facilita la creación y el escalado de soluciones de IoT para una variedad de industrias y casos de uso. 

Los usuarios utilizarán la plataforma para delimitar las áreas permitidas para la circulación de los animales, establecer las alarmas y monitorear el ganado en tiempo real. Además, se establecerá la conexión de la aplicación generada en \emph{AllThingsTalk} a una base de datos donde se guardará el registro médico de cada animal. 


\section{2. Identificación y análisis de los interesados}
\label{sec:interesados}

\begin{table}[ht]
%\caption{Identificación de los interesados}
%\label{tab:interesados}
\begin{tabularx}{\linewidth}{@{}|l|X|X|l|@{}}
\hline
\rowcolor[HTML]{C0C0C0} 
Rol           & Nombre y Apellido & Organización 	& Puesto 	\\ \hline
Cliente       & \clientename      &\empclientename	&   -     	\\ \hline
Responsable   & \authorname       & FIUBA        	& Alumno 	\\ \hline
Orientador    & \supname	      & \pertesupname 	& Director trabajo final \\ \hline
Usuario final & Personal rural         &  -           	&   -     	\\ \hline
\end{tabularx}
\end{table}

 
Características de los interesados:
\begin{itemize}
	\item Cliente: es el propietario del ganado, ya sea una persona o una empresa que busca mejorar la gestión de sus animales mediante el uso de tecnología IoT. Este grupo de interesados espera obtener beneficios como la reducción de costos, la mejora en la calidad de vida de los animales y la disminución de pérdidas o daños causados por el ganado.
	\item Orientador: es responsable de garantizar que el proyecto cumpla con los estándares de calidad y que se sigan las mejores prácticas en el desarrollo de software y hardware. La disponibilidad horaria esta acotada de lunes a viernes de 9 a 18 hs. 
	\item Usuarios: es el personal rural encargado de cuidar y monitorear el ganado utilizando el sistema de gestión de ganado IoT.
\end{itemize}


\section{3. Propósito del proyecto}
\label{sec:proposito}

El propósito de este proyecto es desarrollar un sistema de monitoreo remoto de ganado que permita a las empresas, o personas relacionadas a este sector, tener acceso a información en tiempo real sobre la ubicación y el estado de su ganado, lo que les permitirá, en base a información detallada, mejorar la toma de decisiones y la eficiencia en la gestión. Este sistema deberá ser fácil de usar y escalable para adaptarse a cualquier tipo de establecimiento, ya sea pequeño o grande.

\section{4. Alcance del proyecto}
\label{sec:alcance}

El alcance del proyecto incluye el desarrollo de un prototipo de sistema de gestión de ganado utilizando tecnología IoT. El sistema estará compuesto de \emph{TAGs} que se fijarán en las orejas de los animales utilizando dispositivos de bajo consumo de energía con tecnología de red de largo alcance conocida como LoRaWAN. Además, se dispondrá de una plataforma en la nube llamada \emph{The Things Stack} para la gestión, el control, almacenamiento y visualización de los datos que generan estos dispositivos. Se integrará la plataforma mediante HTTP a \emph{AllThingsTalk}, que será utilizada por los usuarios para delimitar las áreas permitidas para la circulación de los animales, establecer alarmas y monitorear el ganado en tiempo real. También se establecerá la conexión de la aplicación generada en \emph{AllThingsTalk} a una base de datos donde se guardará el registro médico de cada animal.

No se incluyen en este proyecto aspectos relacionados con el desarrollo de hardware o software para dispositivos móviles. Tampoco se incluye la instalación de la infraestructura necesaria para la conexión a internet o para el uso del sistema de gestión de ganado. Asimismo, este proyecto no contempla la producción a gran escala de los \emph{TAGs} o la distribución de los mismos. 

Este proyecto se centrará únicamente en el desarrollo de un prototipo que pueda ser utilizado como una prueba de concepto para demostrar la viabilidad del sistema de gestión de ganado mediante IoT.


\section{5. Supuestos del proyecto}
\label{sec:supuestos}

Para el desarrollo del presente proyecto se supone que:

\begin{itemize}
	\item Se contará con el tiempo necesario para la realización de las tareas específicas del proyecto.
	\item Los recursos financieros necesarios para la ejecución del proyecto estarán disponibles en tiempo y forma.
	\item La tecnología y las herramientas necesarias para el desarrollo del proyecto estarán disponibles y en condiciones óptimas de funcionamiento.
	\item No habrá cambios significativos en las condiciones macroeconómicas que afecten de manera importante la ejecución del proyecto.
	\item El entorno regulatorio y normativo en el que se desarrollará el proyecto permanecerá estable y sin cambios importantes que afecten la ejecución del proyecto.
	\item Los proveedores y terceros involucrados en el proyecto cumplirán con sus compromisos en tiempo y forma.
	\item Los animales se encuentran en un radio igual o menor en el que se puede utilizar la tecnología de LoRaWAN. 
	\item Los imprevistos y cambios inesperados que puedan surgir durante el desarrollo del proyecto se abordarán de manera eficiente y efectiva para minimizar su impacto en el éxito del proyecto.
	\item Se mantendrá una comunicación clara y efectiva con los interesados involucrados en el proyecto para garantizar una adecuada coordinación y colaboración.
\end{itemize}

\vspace*{\fill}
\pagebreak

\section{6. Requerimientos}
\label{sec:requerimientos}

\begin{enumerate}
	\item Requerimientos funcionales
		\begin{enumerate}
			\item El sistema debe permitir la instalación y configuración de TAGs electrónicas en las orejas de los animales.
			\item Debe utilizar dispositivos de bajo consumo de energía con tecnología LoRaWAN para la transmisión de datos.
			\item debe recopilar y procesar la información de las TAGs electrónicas y enviarla al gateway.
			\item El sistema debe proporcionar una plataforma para la gestión, control, almacenamiento y visualización de los datos generados por los dispositivos.
			\item El sistema debe permitir delimitar áreas permitidas para la circulación de los animales.
			\item El sistema debe ofrecer la posibilidad de establecer alarmas para eventos relevantes, como la salida de un animal del área permitida.
			\item El sistema debe permitir el monitoreo en tiempo real del ganado a través de un tablero personalizable.
		\end{enumerate}
	\item Requerimientos de seguridad
		\begin{enumerate}
			\item El sistema debe garantizar la privacidad y seguridad de los datos de IoT mediante características de seguridad avanzadas, como el cifrado de extremo a extremo y protocolos de comunicación seguros.
			\item El sistema debe contar con un proceso de autenticación seguro para nuevos dispositivos que se conecten a la red.
		\end{enumerate}
		
	\item Requerimientos de integración
		\begin{enumerate}
			\item El sistema debe integrarse con la plataforma en la nube \emph{The Things Stack} a través de una conexión a internet.
			\item El sistema debe permitir la integración con otras plataformas en la nube, bases de datos y herramientas de análisis mediante HTTP, utilizando la plataforma \emph{AllThingsTalk}.
		\end{enumerate}
		
	\item Requerimientos de documentación
		\begin{enumerate}
			\item Se debe generar documentación detallada sobre la instalación, configuración y uso del sistema.
			\item Se debe proporcionar documentación sobre las normas y regulaciones vigentes relacionadas con el monitoreo remoto de ganado
		\end{enumerate}
		
	\item Requerimiento de testing
		\begin{enumerate}
			\item El sistema debe ser sometido a pruebas exhaustivas para verificar su funcionalidad y rendimiento.
			\item Se deben realizar pruebas de integración para garantizar la correcta comunicación entre los componentes del sistema.
			\item Se deben realizar pruebas de interoperabilidad para asegurar que el sistema funcione correctamente con diferentes dispositivos y plataformas.
			\item Se deben realizar pruebas de seguridad para identificar posibles vulnerabilidades y asegurar la protección de los datos.
			\item Se deben documentar los resultados de las pruebas y realizar las correcciones necesarias antes de la implementación final del sistema.
		\end{enumerate}
		
	\item Requerimientos de la interfaz
		\begin{enumerate}			
			\item La interfaz de usuario debe ser intuitiva y fácil de usar, permitiendo a los usuarios interactuar de manera sencilla con el sistema.
			\item Debe existir un panel de control centralizado que muestre información en tiempo real sobre el estado del ganado, incluyendo ubicación, actividad y alertas.
			\item  Se deben proporcionar herramientas de configuración y personalización de las áreas permitidas para la circulación del ganado, así como la definición de alarmas y notificaciones.
			\item La interfaz debe permitir el acceso a un registro médico completo de cada animal, incluyendo historial de enfermedades, vacunas y tratamientos.
			\item La interfaz debe ser compatible con diferentes navegadores web populares, como Chrome, Firefox, Safari, etc.
			\item Se debe garantizar la seguridad de la interfaz, protegiendo los datos y proporcionando mecanismos de autenticación y autorización para los usuarios.
		\end{enumerate}
		
\end{enumerate}

\begin{consigna}{red}
Leyendo los requerimientos se debe poder interpretar cómo será el proyecto y su funcionalidad.

Indicar claramente cuál es la prioridad entre los distintos requerimientos y si hay requerimientos opcionales. 

No olvidarse de que los requerimientos incluyen a las regulaciones y normas vigentes!!!

Y al escribirlos seguir las siguientes reglas:
\begin{itemize}
	\item Ser breve y conciso (nadie lee cosas largas). 
	\item Ser específico: no dejar lugar a confusiones.
	\item Expresar los requerimientos en términos que sean cuantificables y medibles.
\end{itemize}
\end{consigna}


\section{7. Historias de usuarios (\textit{Product backlog})}
\label{sec:backlog}

\begin{consigna}{red}
Descripción: En esta sección se deben incluir las historias de usuarios y su ponderación (\textit{history points}). Recordar que las historias de usuarios son descripciones cortas y simples de una característica contada desde la perspectiva de la persona que desea la nueva capacidad, generalmente un usuario o cliente del sistema. La ponderación es un número entero que representa el tamaño de la historia comparada con otras historias de similar tipo.

El formato propuesto es: "como [rol] quiero [tal cosa] para [tal otra cosa]."

Se debe indicar explícitamente el criterio para calcular los \textit{story points} de cada historia
\end{consigna}

\section{8. Entregables principales del proyecto}
\label{sec:entregables}

\begin{consigna}{red}

Los entregables del proyecto son (ejemplo):

\begin{itemize}
	\item Manual de uso
	\item Diagrama de circuitos esquemáticos
	\item Código fuente del firmware
	\item Diagrama de instalación
	\item Informe final
	\item etc...
\end{itemize}

\end{consigna}

\section{9. Desglose del trabajo en tareas}
\label{sec:wbs}

\begin{consigna}{red}
El WBS debe tener relación directa o indirecta con los requerimientos.  Son todas las actividades que se harán en el proyecto para dar cumplimiento a los requerimientos. Se recomienda mostrar el WBS mediante una lista indexada:

\begin{enumerate}
\item Grupo de tareas 1
	\begin{enumerate}
	\item Tarea 1 (tantas h)
	\item Tarea 2 (tantas hs)
	\item Tarea 3 (tantas h)
	\end{enumerate}
\item Grupo de tareas 2
	\begin{enumerate}
	\item Tarea 1 (tantas h)
	\item Tarea 2 (tantas h)
	\item Tarea 3 (tantas h)
	\end{enumerate}
\item Grupo de tareas 3
	\begin{enumerate}
	\item Tarea 1 (tantas h)
	\item Tarea 2 (tantas h)
	\item Tarea 3 (tantas h)
	\item Tarea 4 (tantas h)
	\item Tarea 5 (tantas h)
	\end{enumerate}
\end{enumerate}

Cantidad total de horas: (tantas h)

Se recomienda que no haya ninguna tarea que lleve más de 40 h. 

\end{consigna}

\section{10. Diagrama de Activity On Node}
\label{sec:AoN}

\begin{consigna}{red}
Armar el AoN a partir del WBS definido en la etapa anterior. 

%La figura \ref{fig:AoN} fue elaborada con el paquete latex tikz y pueden consultar la siguiente referencia \textit{online}:

%\url{https://www.overleaf.com/learn/latex/LaTeX_Graphics_using_TikZ:_A_Tutorial_for_Beginners_(Part_3)\%E2\%80\%94Creating_Flowcharts}

\end{consigna}

\begin{figure}[htpb]
\centering 
\includegraphics[width=.8\textwidth]{./Figuras/AoN.png}
\caption{Diagrama de \textit{Activity on Node}.}
\label{fig:AoN}
\end{figure}

Indicar claramente en qué unidades están expresados los tiempos.
De ser necesario indicar los caminos semicríticos y analizar sus tiempos mediante un cuadro.
Es recomendable usar colores y un cuadro indicativo describiendo qué representa cada color, como se muestra en el siguiente ejemplo:



\section{11. Diagrama de Gantt}
\label{sec:gantt}

\begin{consigna}{red}

Existen muchos programas y recursos \textit{online} para hacer diagramas de Gantt, entre los cuales destacamos:

\begin{itemize}
\item Planner
\item GanttProject
\item Trello + \textit{plugins}. En el siguiente link hay un tutorial oficial: \\ \url{https://blog.trello.com/es/diagrama-de-gantt-de-un-proyecto}
\item Creately, herramienta online colaborativa. \\\url{https://creately.com/diagram/example/ieb3p3ml/LaTeX}
\item Se puede hacer en latex con el paquete \textit{pgfgantt}\\ \url{http://ctan.dcc.uchile.cl/graphics/pgf/contrib/pgfgantt/pgfgantt.pdf}
\end{itemize}

Pegar acá una captura de pantalla del diagrama de Gantt, cuidando que la letra sea suficientemente grande como para ser legible. 
Si el diagrama queda demasiado ancho, se puede pegar primero la ``tabla'' del Gantt y luego pegar la parte del diagrama de barras del diagrama de Gantt.

Configurar el software para que en la parte de la tabla muestre los códigos del EDT (WBS).\\
Configurar el software para que al lado de cada barra muestre el nombre de cada tarea.\\
Revisar que la fecha de finalización coincida con lo indicado en el Acta Constitutiva.

En la figura \ref{fig:gantt}, se muestra un ejemplo de diagrama de Gantt realizado con el paquete de \textit{pgfgantt}. En la plantilla pueden ver el código que lo genera y usarlo de base para construir el propio.

\begin{figure}[htbp]
\begin{center}
\begin{ganttchart}{1}{12}
  \gantttitle{2020}{12} \\
  \gantttitlelist{1,...,12}{1} \\
  \ganttgroup{Group 1}{1}{7} \\
  \ganttbar{Task 1}{1}{2} \\
  \ganttlinkedbar{Task 2}{3}{7} \ganttnewline
  \ganttmilestone{Milestone o hito}{7} \ganttnewline
  \ganttbar{Final Task}{8}{12}
  \ganttlink{elem2}{elem3}
  \ganttlink{elem3}{elem4}
\end{ganttchart}
\end{center}
\caption{Diagrama de Gantt de ejemplo}
\label{fig:gantt}
\end{figure}


\begin{landscape}
\begin{figure}[htpb]
\centering 
\includegraphics[height=.85\textheight]{./Figuras/Gantt-2.png}
\caption{Ejemplo de diagrama de Gantt rotado}
\label{fig:diagGantt}
\end{figure}

\end{landscape}

\end{consigna}


\section{12. Presupuesto detallado del proyecto}
\label{sec:presupuesto}

\begin{consigna}{red}
Si el proyecto es complejo entonces separarlo en partes:
\begin{itemize}
	\item Un total global, indicando el subtotal acumulado por cada una de las áreas.
	\item El desglose detallado del subtotal de cada una de las áreas.
\end{itemize}

IMPORTANTE: No olvidarse de considerar los COSTOS INDIRECTOS.

\end{consigna}

\begin{table}[htpb]
\centering
\begin{tabularx}{\linewidth}{@{}|X|c|r|r|@{}}
\hline
\rowcolor[HTML]{C0C0C0} 
\multicolumn{4}{|c|}{\cellcolor[HTML]{C0C0C0}COSTOS DIRECTOS} \\ \hline
\rowcolor[HTML]{C0C0C0} 
Descripción &
  \multicolumn{1}{c|}{\cellcolor[HTML]{C0C0C0}Cantidad} &
  \multicolumn{1}{c|}{\cellcolor[HTML]{C0C0C0}Valor unitario} &
  \multicolumn{1}{c|}{\cellcolor[HTML]{C0C0C0}Valor total} \\ \hline
 &
  \multicolumn{1}{c|}{} &
  \multicolumn{1}{c|}{} &
  \multicolumn{1}{c|}{} \\ \hline
 &
  \multicolumn{1}{c|}{} &
  \multicolumn{1}{c|}{} &
  \multicolumn{1}{c|}{} \\ \hline
\multicolumn{1}{|l|}{} &
   &
   &
   \\ \hline
\multicolumn{1}{|l|}{} &
   &
   &
   \\ \hline
\multicolumn{3}{|c|}{SUBTOTAL} &
  \multicolumn{1}{c|}{} \\ \hline
\rowcolor[HTML]{C0C0C0} 
\multicolumn{4}{|c|}{\cellcolor[HTML]{C0C0C0}COSTOS INDIRECTOS} \\ \hline
\rowcolor[HTML]{C0C0C0} 
Descripción &
  \multicolumn{1}{c|}{\cellcolor[HTML]{C0C0C0}Cantidad} &
  \multicolumn{1}{c|}{\cellcolor[HTML]{C0C0C0}Valor unitario} &
  \multicolumn{1}{c|}{\cellcolor[HTML]{C0C0C0}Valor total} \\ \hline
\multicolumn{1}{|l|}{} &
   &
   &
   \\ \hline
\multicolumn{1}{|l|}{} &
   &
   &
   \\ \hline
\multicolumn{1}{|l|}{} &
   &
   &
   \\ \hline
\multicolumn{3}{|c|}{SUBTOTAL} &
  \multicolumn{1}{c|}{} \\ \hline
\rowcolor[HTML]{C0C0C0}
\multicolumn{3}{|c|}{TOTAL} &
   \\ \hline
\end{tabularx}%
\end{table}


\section{13. Gestión de riesgos}
\label{sec:riesgos}

\begin{consigna}{red}
a) Identificación de los riesgos (al menos cinco) y estimación de sus consecuencias:
 
Riesgo 1: detallar el riesgo (riesgo es algo que si ocurre altera los planes previstos de forma negativa)
\begin{itemize}
	\item Severidad (S): mientras más severo, más alto es el número (usar números del 1 al 10).\\
	Justificar el motivo por el cual se asigna determinado número de severidad (S).
	\item Probabilidad de ocurrencia (O): mientras más probable, más alto es el número (usar del 1 al 10).\\
	Justificar el motivo por el cual se asigna determinado número de (O). 
\end{itemize}   

Riesgo 2:
\begin{itemize}
	\item Severidad (S): 
	\item Ocurrencia (O):
\end{itemize}

Riesgo 3:
\begin{itemize}
	\item Severidad (S): 
	\item Ocurrencia (O):
\end{itemize}


b) Tabla de gestión de riesgos:      (El RPN se calcula como RPN=SxO)

\begin{table}[htpb]
\centering
\begin{tabularx}{\linewidth}{@{}|X|c|c|c|c|c|c|@{}}
\hline
\rowcolor[HTML]{C0C0C0} 
Riesgo & S & O & RPN & S* & O* & RPN* \\ \hline
       &   &   &     &    &    &      \\ \hline
       &   &   &     &    &    &      \\ \hline
       &   &   &     &    &    &      \\ \hline
       &   &   &     &    &    &      \\ \hline
       &   &   &     &    &    &      \\ \hline
\end{tabularx}%
\end{table}

Criterio adoptado: 
Se tomarán medidas de mitigación en los riesgos cuyos números de RPN sean mayores a...

Nota: los valores marcados con (*) en la tabla corresponden luego de haber aplicado la mitigación.

c) Plan de mitigación de los riesgos que originalmente excedían el RPN máximo establecido:
 
Riesgo 1: plan de mitigación (si por el RPN fuera necesario elaborar un plan de mitigación).
  Nueva asignación de S y O, con su respectiva justificación:
  - Severidad (S): mientras más severo, más alto es el número (usar números del 1 al 10).
          Justificar el motivo por el cual se asigna determinado número de severidad (S).
  - Probabilidad de ocurrencia (O): mientras más probable, más alto es el número (usar del 1 al 10).
          Justificar el motivo por el cual se asigna determinado número de (O).

Riesgo 2: plan de mitigación (si por el RPN fuera necesario elaborar un plan de mitigación).
 
Riesgo 3: plan de mitigación (si por el RPN fuera necesario elaborar un plan de mitigación).

\end{consigna}


\section{14. Gestión de la calidad}
\label{sec:calidad}

\begin{consigna}{red}
Elija al menos diez requerientos que a su criterio sean los más importantes/críticos/que aportan más valor y para cada uno de ellos indique las acciones de verificación y validación que permitan asegurar su cumplimiento.

\begin{itemize} 
\item Req \#1: copiar acá el requerimiento.

\begin{itemize}
	\item Verificación para confirmar si se cumplió con lo requerido antes de mostrar el sistema al cliente. Detallar 
	\item Validación con el cliente para confirmar que está de acuerdo en que se cumplió con lo requerido. Detallar  
\end{itemize}

\end{itemize}

Tener en cuenta que en este contexto se pueden mencionar simulaciones, cálculos, revisión de hojas de datos, consulta con expertos, mediciones, etc.  Las acciones de verificación suelen considerar al entregable como ``caja blanca'', es decir se conoce en profundidad su funcionamiento interno.  En cambio, las acciones de validación suelen considerar al entregable como ``caja negra'', es decir, que no se conocen los detalles de su funcionamiento interno.

\end{consigna}

\section{15. Procesos de cierre}    
\label{sec:cierre}

\begin{consigna}{red}
Establecer las pautas de trabajo para realizar una reunión final de evaluación del proyecto, tal que contemple las siguientes actividades:

\begin{itemize}
	\item Pautas de trabajo que se seguirán para analizar si se respetó el Plan de Proyecto original:
	 - Indicar quién se ocupará de hacer esto y cuál será el procedimiento a aplicar. 
	\item Identificación de las técnicas y procedimientos útiles e inútiles que se emplearon, y los problemas que surgieron y cómo se solucionaron:
	 - Indicar quién se ocupará de hacer esto y cuál será el procedimiento para dejar registro.
	\item Indicar quién organizará el acto de agradecimiento a todos los interesados, y en especial al equipo de trabajo y colaboradores:
	  - Indicar esto y quién financiará los gastos correspondientes.
\end{itemize}

\end{consigna}


\end{document}
